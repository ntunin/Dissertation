\chapter{Дифференциальная эволюция}\label{sec:radio}
// TODO: Подробнее о дифф. эволюции
Как было показано выше, градиентный метод может предоставить решение, не являющийся локальным оптимумом, если целевая функция слабо изменяется в окрестности текущей точки. Алгоритм дифференциальной эволюции не подвержен такому поведению.
Эксперименты показывают, что в целом эволюция популяции соответствует динамике случайного облака точек, движущегося как целое вдоль рельефа оптимизируемой функции, повторяя его характерные особенности. В случае попадания в овраг «облако» принимает форму этого оврага и распределение точек становится таким, что математическое ожидание разности двух случайных векторов оказывается направленным вдоль длинной стороны оврага. Это обеспечивает быстрое движение вдоль узких вытянутых оврагов, тогда как для градиентных методов в аналогичных условиях характерно колебательная динамика «от стенки к стенке». Приведенные эвристические соображения иллюстрируют наиболее важную и привлекательную особенность алгоритма ДЭ — способность динамически моделировать особенности рельефа оптимизируемой функции. Именно этим объясняется замечательная способность алгоритма быстро проходить сложные овраги, обеспечивая эффективность даже в случае сложного рельефа.

Кратко опишем идею алгоритма ДЭ. В начале происходит генерация популяции. Если нет дополнительной информации, такая популяция особи популяции генерируются случайным образом с равномерным распределением. Затем, пока все особи не сойдутся в одной точке, каждая особь подвергается мутации путем присваивания ей признаков другой особи. Процедура, определяющая, в какой степени признаки других особей участвуют в эволюции конкретной особи является параметром алгоритма. Далее происходит сравнение значений целевой функции мутировавшей особи со значением целевой функции исходной особи. Выживает особь с лучшим значением целевой функции.

Данный алгоритм хорошо поддается модификации для запуска в параллельных потоках. В этом случае, на очередной итерации алгоритма выбирается некоторый набор особей, эволюция каждой из которых на данной итерации происходит независимо от эволюции другой. В данной работе был предложен вариант реализации алгоритма дифференциальной эволюции, адаптированный для запуска на графическом устройстве. Использование алгоритма ДЭ позволило достичь решения с целевой функцией $\tilde{F} = 253$ для задачи СВД'~2x2.

