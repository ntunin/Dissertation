\begin{frame}[noframenumbering,plain]
    \setcounter{framenumber}{1}
    \maketitle
\end{frame}

\begin{frame}
    \frametitle{Положения, выносимые на защиту}
    \begin{itemize}
        \item В задаче имеется одномерная непрерывная группа симметрий.
        \item Имеются конфигурации ФАР, при которых учет взаимного влияния излучателей ведет к существенному увеличению коэффициента усиления в заданном направлении.
        \item Для многих конфигураций ФАР задача имеет несколько кластеров из локальных оптимумов с одинаковым значением целевой функции, не эквивалентных относительно равного сдвига фаз во всех излучателях.
        \item Использование градиентных методов в комбинации с методом ветвей и границ позволяет достичь более качественных решений по сравнению с алгоритмом дифференциальной эволюции в задаче оптимизации фаз и амплитуд ФАР.
    \end{itemize}
\end{frame}
\note{
    Проговариваются вслух положения, выносимые на защиту
}

\begin{frame}
    \frametitle{Содержание}
    \tableofcontents
\end{frame}
\note{
    Работа состоит из трех глав.

    \medskip
    Первая глава посвящена обзору имеющейся литературы и постановке задачи математического программирования.

    Вторая глава посвящена исследованию структуры локальных оптимумов с помощью различных алгоритмов оптимизации.

    Третья глава посвящена исследованию ФАР в различных условиях.
}
