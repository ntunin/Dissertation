\begin{frame}[noframenumbering,plain]
    \setcounter{framenumber}{1}
    \maketitle
\end{frame}

\begin{frame}
    \frametitle{Актуальность}
    Будучи собранными в антенную систему, излучатели формируют диаграмму направленности, которая зависит от их расположения и конструкции, а также выбора фаз и амплитуд сигналов, подаваемых на вход излучателей. Возможность формирования направленного излучения позволяет достичь увеличения дальности радиосвязи или уменьшить затраты ресурсов (энергии, площади, занимаемой антеннами, стоимости оборудования и др.).
\end{frame}
\note{
    В настоящее время разработка и анализ эффективных систем радиосвязи имеет большое значение для народного хозяйства. Одной из актуальных задач в этой области является задача оптимизации направленности фазированных антенных решеток (ФАР), представляющих собой антенные системы, распределение фаз и амплитуд на элементах которых позволяет получать направленное излучение. Будучи собранными в антенную систему, излучатели формируют диаграмму направленности, которая зависит от их расположения и конструкции, а также выбора фаз и амплитуд сигналов, подаваемых на вход излучателей. Возможность формирования направленного излучения позволяет достичь увеличения дальности радиосвязи или уменьшить затраты ресурсов (энергии, площади, занимаемой антеннами, стоимости оборудования и др.). 
}

\begin{frame}
    \frametitle{Состояние исследований}
    В СВЧ диапазоне задачи оптимизации направленности ФАР решаются с использованием упрощающающих предположений~(М.~Инденбом, В.~Ижуткин и др,~2018; С.~Щелкунов, Г.~Фриис,~1952; И.А.~Фаняев, В.П.~Кудин,~2017).

    \vspace{1em}
    В ВЧ диапазоне требуется использовать методы электродинамического моделирования для оценки наведенных токов в каждой паре излучателей и учитывать свойства подстилающей поверхности, в результате задача оптимизации направленности ФАР оказывается более сложной, и потому менее изучена~(В.П.~Кудин,~2014; А.С.~Юрков, 2016).
    \vspace{1em}

    Существует постановка задачи при ограничении суммарной мощности, подаваемой на антенную систему. Такая задача может быть решена аналитически~(А.С.~Юрков,~2014).
\end{frame}
\note{
В диапазоне сверхвысоких частот задачи оптимизации фаз и амплитуд излучателелей, как правило, решаются с использованием некоторых упрощающающих предположений. Однако, в диапазоне высоких частот задача оптимизации направленности ФАР оказывается более сложной, и потому менее изучена. При ограничении суммарной мощности, подаваемой на антенную систему, задача выбора фаз и амплитуд на излучателях может быть решена аналитически. Однако, при ограничении на мощность по каждому входу антенной системы требуется решение невыпуклых задач квадратичного программирования, которые, вообще говоря, являются NP-трудными.
}

\begin{frame}
    \frametitle{Цели и задачи}
    \textbf{Цель:} Создание алгоритмов оптимизации направленности излучения ФАР и исследование области применимости различных методов решения этой задачи.
    
    \textbf{Задачи}:
    \begin{enumerate}
      \item Изучить структуру множества локальных оптимумов и наличие симметрий в рассматриваемой задаче.
      \item Разработать алгоритмы решения задачи, учитывающие структуру множества локальных оптимумов и использующие известные методы математической оптимизации.
      \item Сравнить в вычислительных экспериментах предложенные алгоритмы с известными
      \item Исследовать на основе вычислительных экспериментов влияние расположения излучателей и используемой радиочастоты на эффективность работы алгоритмов оптимизации
      \item Сравнить коэффициент усиления ФАР при оптимизации направленности излучения с учетом взаимного влияния излучателей и без учета этого фактора
    \end{enumerate}
\end{frame}
\note{
Целью данной работы является создание алгоритмов оптимизации направленности излучения ФАР и исследование области применимости различных методов решения этой задачи. Для достижения указанной цели были решены следующие задачи:
Изучить структуру множества локальных оптимумов и наличие симметрий в рассматриваемой задаче.
Разработать алгоритмы решения задачи, учитывающие структуру множества локальных оптимумов и использующие известные методы математической оптимизации.
Сравнить в вычислительных экспериментах предложенные алгоритмы с известными
Исследовать на основе вычислительных экспериментов влияние расположения излучателей и используемой радиочастоты на эффективность работы алгоритмов оптимизации
Сравнить коэффициент усиления ФАР при оптимизации направленности излучения с учетом взаимного влияния излучателей и без учета этого фактора
}

\begin{frame}
    \frametitle{Методология и методы исследования}
    Методы градиентной оптимизации, эволюционные алгоритмы, проведение вычислительного эксперимента, методы математической статистики, использование линейных симметрий задачи
\end{frame}
\note{
В данной работе рассматривается подход к решению задачи максимизации направленности излучения ФАР в заданном направлении при ограничениях, накладываемых на мощность, подаваемую на каждый из излучателей. Такая задача может быть решена только численными методами. Для использования градиентного метода задача сводится к задаче безусловной оптимизации методом штрафных функций.

Вообще говоря, при использовании метода градиентного подъема не гарантируется получение глобального оптимума. Приблизиться к глобальному оптимуму позволяет многократный запуск алгоритма из случайным образом сгенерированных точек. Кроме того, многократный запуск позволяет оценить количество локальных оптимумов, что является некоторым критерием сложности индивидуальной задачи. Анализ структуры
локальных оптимумов позволяет также выявить наличие нетривиальных симметрий.

Еще одним широко используемым подоходом к решению задач оптимизации ФАР являются эволюционные алгоритмы, и, в частности, генетические алгоритмы, роевые алгоритмы, алгоритмы дифференциальной эволюции. Использование эволюционных методов требует больше времени, чем использование градиентного подъема, однако, в отличие от градиентных методов, не требует вычисления производных и не подвержен преждевременному завершению в точках стационарности.

Для оценки качества результатов градиентного алгоритма производится их сравнение с решениями, полученными с помощью решателя BARON в пакете GAMS. BARON использует алгоритмы метода ветвей и границ, усиленные различными методами распространения ограничений и двойственности для уменьшения диапазонов переменных в ходе работы алгоритма. Его использование также представляет альтернативный подход к решению данной задачи, но, поскольку BARON является коммерческим решателем, произведение расчетов требует приобретения лицензии, что не всегда приемлемо.
}

\begin{frame}
    \frametitle{Содержание}
    \tableofcontents
\end{frame}
\note{
    Работа состоит из четырех глав.

    \medskip
    Первая глава посвящена посвящена обзору имеющейся литературы и постановке задачи в виде задачи математического программирования. Здесь рассматриваются источники, позволяющие сформулировать постановку задачи, приводится анализ аналогичных исследований. Также, в рамках этого раздела производится сравнение результатов работы различных методов решения поставленной задачи.
    
    Во второй главе производится исследование возможности оптимизации поставленной задачи методами дифференциальной эволюции (ДЭ).
    
    Третья глава посвящена исследованию структуры локальных оптимумов с помощью различных алгоритмов оптимизации, производится анализ наличия групп непрерывных симметрий.

    Четвертая глава посвящена исследованию возможности оптимизации возбуждения ФАР в различных условиях.
}
