\begin{frame}[noframenumbering,plain]
    \setcounter{framenumber}{1}
    \maketitle
\end{frame}

\begin{frame}
    \frametitle{Актуальность}
     В настоящее время разработка и анализ эффективных систем радиосвязи имеет большое значение для народного хозяйства. Одной из актуальных задач в этой области является задача оптимизации направленности фазированных антенных решеток (ФАР), представляющих собой антенные системы, распределение фаз и амплитуд на излучателях которых позволяет получать направленное излучение.
    
    \vspace{1em}


    Будучи собранными в антенную систему и разведенными в пространстве, излучатели формируют диаграмму направленности, которая зависит от расположения и конструкции излучателей, а также выбора фаз и амплитуд сигналов, подаваемых на вход излучателей.
\end{frame}
\note{
    Проговариваются вслух положения, выносимые на защиту
}

\begin{frame}
    \frametitle{Состояние исследований}
    В СВЧ диапазоне задачи оптимизации направленности ФАР решаются с использованием упрощающающих предположений~(М.~Инденбом, В.~Ижуткин и др,~2018; С.~Щелкунов, Г.~Фриис,~1952; И.А.~Фаняев, В.П.~Кудин,~2017).
    
    \vspace{1em}
    В ВЧ диапазоне требуется использовать методы электродинамического моделирования для оценки наведенных токов в каждой паре излучателей и учитывать свойства подстилающей поверхности, в результате задача оптимизации направленности ФАР оказывается более сложной, и потому менее изучена~(В.П.~Кудин,~2014; А.С.~Юрков, 2016).
    \vspace{1em}
    
    Существует постановка задачи при ограничении суммарной мощности, подаваемой на антенную систему. Такая задача может быть решена аналитически~(А.С.~Юрков,~2014).
\end{frame}
\note{
}

\begin{frame}
    \frametitle{Содержание}
    \tableofcontents
\end{frame}
\note{
    Работа состоит из трех глав.

    \medskip
    Первая глава посвящена обзору имеющейся литературы и постановке задачи математического программирования.

    Вторая глава посвящена исследованию структуры локальных оптимумов с помощью различных алгоритмов оптимизации.

    Третья глава посвящена исследованию ФАР в различных условиях.
}
