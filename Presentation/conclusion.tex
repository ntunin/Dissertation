\begin{frame}
    \frametitle{Научная новизна}
    \begin{itemize}

    \item В настоящей работе максимизируется излучение в одном заданном направлении, что в наибольшей степени соответствует системам коротковолновой радиосвязи.
    \item Разработаны реализации алгоритмов с процедурой возвращения в допустимую область.
    \item Обоснован подход к снижению размерности задачи фиксированием значения переменной.
    \item Впервые показано наличие кластеров из локальных оптимумов.
    \item На открытый вопрос о целесообразности учета взаимного влияния излучателей при оптимизации направленности ФАР КВ диапазона дан положительный ответ.
    \end{itemize}
\end{frame}
\note{
    Проговаривается вслух научная новизна
}

\begin{frame}
    \frametitle{Научная и практическая значимость}
    \begin{itemize}
        \item Разработанные алгоритмы оптимизации возбуждения ФАР могут  применяться в системах связи коротковолнового диапазона для увеличения дальности, снижения энергозатрат или площади, занимаемой антеннами. Созданное программное обеспечение позволяет производить необходимые для этого расчеты.
        \item Полученное обоснование необходимости учета взаимного влияния излучателей при оптимизации напраленности ФАР, а также результаты вычислительных экспериментов для различных вариантов ФАР могут быть полезны при проектировании новых антенных систем.

    \end{itemize}
\end{frame}
\note{
    Проговариваются вслух научная и практическая значимость
}

%\begin{frame}
%    \frametitle{Свидетельство о регистрации программы}
%    \begin{figure}[h]
%        \centering
%        \includegraphics[height=0.7\textheight]{registration}
%    \end{figure}
%\end{frame}
%\note{
%    Получено свидетельство о регистрации разработанной программы \textsc{Hello~world™}.
%}

%\begin{frame}
%    \frametitle{Акт о внедрении}
%    \begin{figure}[h]
%        \centering
%        \fbox{
%            \begin{minipage}[t]{0.4\linewidth}
%                \includegraphics[width=\linewidth]{implementation}
%            \end{minipage}
%        }
%    \end{figure}
%\end{frame}
%\note{
%    Получен акт о внедрении.
%}

%\begin{frame} % публикации на одной странице
 \begin{frame}[t,allowframebreaks] % публикации на нескольких страницах
    \frametitle{Основные публикации}
    \nocite{tyu:daor}%
    \nocite{tyu22:reis}%
    \nocite{tyu:jphys}%
    \nocite{tyu:motor}%
    \ifnumequal{\value{bibliosel}}{0}{
        \insertbiblioauthor
    }{
        \printbibliography%
    }
\end{frame}
\note{
    Результаты работы опубликованы в N печатных изданиях,
    в~т.\:ч. M реферируемых изданиях.
}

\begin{frame}
    \frametitle{Апробация}
    \begin{itemize}
      \item VII Международной конференции «Проблемы оптимизации и их приложения» - Омск, июль 2018.
      \item Международной конференции «Теория математической оптимизации и исследование операций» - Екатеринбург, июль 2019, Иркутск; июль 2021.
      \item V Международной научно-технической конференции «Радиотехника, электроника и связь» - Омск, октябрь 2019.
      \item Математическое моделирование и дискретная оптимизация (ОФИМ СО РАН~им.~С.Л.~Соболева, ОмГУ им.~Ф.М.~Достоевского).
      \item Перспективы развития радиосвязи и приборостроения (АО <<ОНИИП>>).
      \item Современные проблемы радиофизики и радиотехники (АО <<ОНИИП>>, ОмГУ им.~Ф.М.~Достоевского).
    \end{itemize}
\end{frame}
\note{
    Работа была представлена на ряде конференций.
}

\begin{frame}
    \frametitle{Полученные результаты}
    \begin{enumerate}
\item Предложена модификация градиентного метода, учитывающая специфику задачи оптимизации фаз и амплитуд ФАР, позволяющая получать решения с практически приемлемыми точностью и временем счета.
  \item В рассматриваемых задачах оптимизации фаз и амплитуд ФАР методами линейной алгебры выявлено семейство  симметрий, состоящих в равном по величине сдвиге фаз во всех излучателях и позволяющее сократить размерность задач.
  \item В ходе вычислительного эксперимента показано, что задача оптимизации фаз и амплитуд фазированной антенной решетки имеет многочисленные локальные оптимумы, большое число из которых совпадают по целевой функции, однако не эквивалентны между собой относительно равного сдвига фаз во всех излучателях.
  \item Выявлены ситуации, в которых коэффициент усиления, соответствующий решению задачи квадратичной оптимизации, имеет существенное преимущество перед коэффициентом усиления, получаемым стандартным методом простого фазирования.
\end{enumerate}
\end{frame}

\begin{frame}[plain, noframenumbering] % последний слайд без оформления
    \begin{center}
        \Huge
        Спасибо за внимание!
    \end{center}
\end{frame}
