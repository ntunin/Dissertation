
{\actuality}
\ifnumequal{\value{myspec}}{0}{
    Здесь могла бы быть актуальность для 05.13.18
}{
В настоящее время разработка и анализ эффективных систем радиосвязи имеет большое значение для народного хозяйства. Одной из актуальных задач в этой области является задача оптимизации направленности фазированных антенных решеток (ФАР), представляющих собой антенные системы, распределение фаз и амплитуд на элементах которых позволяет получать направленное излучение.
%Интенсивность излучения каждого отдельно взятого излучателя такой системы зависит от геометрии самого излучателя и радиочастоты сигнала, поданного на его вход. Характеристика интенсивности излучения антенны в пространстве называется диаграммой направленности.
Будучи собранными в антенную систему и разведенными в пространстве, излучатели формируют диаграмму направленности, которая зависит от расположения и конструкции излучателей, а также выбора фаз и амплитуд сигналов, подаваемых на вход излучателей. Возможность формирования направленного излучения позволяет достичь увеличения дальности передачи антенной системы и уменьшить энергозатраты на передачу сигнала.
%Это связано с тем, что часть излучения итерферирует с собственным излучением других излучателей, а часть отражается от них и также вносит вклад в результирующее излучение. Таким образом, фазируя определенным образом сигналы, подаваемые на вход каждого излучателя, можно получить максимум интенсивности излучения в заданном направлении.

В диапазоне сверхвысоких частот (СВЧ) задачи оптимизации фаз и амплитуд излучателелей, как правило, решаются с использованием некоторых упрощающающих предположений\mycite{indenbometal:synthesis, schelkunov:antenny, fanyaev2017}{Indenbom~M.,~2018, Щелкунов~С.А.,~1955, Фаняев~И.А.,~1955}. Однако, в диапазоне высоких частот (ВЧ)
%, также называемом коротковолновым, требуется использовать методы электродинамического моделирования для оценки наведенных токов в каждой паре излучателей и учитывать свойства подстилающей поверхности, в результате
задача оптимизации направленности ФАР оказывается более сложной, и потому менее изучена\mycite{kudin2014,yurkov:knd}{Фаняев~И.А.,~2014, Юрков~А.С.,~2016}.
При ограничении суммарной мощности, подаваемой на антенную систему, задача выбора фаз и амплитуд на излучателях
может быть решена аналитически\mycite{yurkov:farkv}{Юрков~А.С.,~2014}.
%\cite{yurkov:farkv}.
Однако, при ограничении на мощность по каждому входу антенной системы требуется решение невыпуклых задач квадратичного программирования\mycite{fuchs:application}{Fuchs~B.,~2014}. Вообще говоря, задачи квадратичного программирования являются NP-трудными\mycite{murty:np}{Murty~K.,~1987}. Для решения таких задач могут применяться методы ветвей и границ~\mycite{nechaeva:bandb}{Нечаева~ М.С.,~2000}, отсечений~\mycite{horst:handbook}{Horst~R.,~2013}, DC-программирования~\mycite{strekalovsky:mindc}{Стрекаловский~А.С.,~2003}, полуопределенной релаксации~\mycite{fuchs:application}{Fuchs~B.,~2014}, эволюционных вычислений~\mycite{boriskinetal:efficient, rao:synthesis}{Boriskin~A.V.,~2010, Rao~A.,~2017}, локального поиска~\mycite{kochetov:local}{Кочетов~Ю.А.,~2008} и др.
Для увеличения дальности передачи радиосигнала и сокращения расхода энергии на передающих ФАР требуется провести исследование свойств задачи оптимизации направленного излучения ФАР и разработать алгоритмы ее решения с использованием наиболее адекватных подходов.
}


% {\progress}
% Этот раздел должен быть отдельным структурным элементом по
% ГОСТ, но он, как правило, включается в описание актуальности
% темы. Нужен он отдельным структурынм элемементом или нет ---
% смотрите другие диссертации вашего совета, скорее всего не нужен.

{\aim} данной работы является создание алгоритмов оптимизации направленности излучения ФАР и исследование
области применимости различных методов решения этой задачи.

Для~достижения указанной цели были решены следующие {\tasks}:
\begin{enumerate}[beginpenalty=10000] % https://tex.stackexchange.com/a/476052/104425
  \item Изучить структуру множества локальных оптимумов и наличие симметрий в рассматриваемой задаче.
 %         и влияние погрешностей в исходных данных на ее решение.
  \item Разработать алгоритмы решения задачи, учитывающие структуру множества локальных оптимумов и использующие известные методы математической оптимизации.
  \item Сравнить предложенные алгоритмы с известными методами.
  \item Исследовать на основе вычислительного эксперимента влияние расположения излучателей и используемой радиочастоты на эффективность работы алгоритмов оптимизации.
  \item Сравнить коэффициент усиления ФАР при оптимизации направленности излучения с учетом взаимного влияния излучателей и без учета этого фактора.
\end{enumerate}


{\novelty}
%Использование ФАР широко распространено в диапазоне сверхвысоких частот (СВЧ), поскольку к
%достоинствам данного диапазона относятся малые размеры антенн и более широкая абсолютная полоса частот. Однако, %радиосистемы СВЧ диапазона характеризуются малой дальностью действия. ФАР высокочастотного диапазона (ВЧ), также %называемого коротковолновым (КВ), менее распространены ввиду больших габаритов, однако, возможность увеличения %дальности действия привлекает к ним особый интерес~\cite{kudzinetal:compact,wilensky:highpower,yalanetal:design}.

%Рассматриваемая в статье задача состоит в максимизации направленности излучения ФАР за счет выбора фаз и амплитуд в каждом излучателе. Задача осложняется за счет взаимного влияния излучателей друг на друга. В СВЧ диапазоне такое влияние можно нивелировать с помощью дополнительных устройств~(см.~\cite{sazonov:PAA}, \S~6.7), после чего задача существенно упрощается~\cite{indenbometal:synthesis}. В случае же с ВЧ диапазоном, в котором используются иначе спроектированные излучатели, такое допущение невозможно.

\begin{enumerate}
%  \item В отличие от других работ по оптимизации направленности ФАР, в настоящей работе производится оптимизация в одном заданном направлении, в связи с эти были созданы новые алгоритмы оптимизации, учитывающие специфику решаемой задачи.
%  \item Предложенный вариант алгоритма градиентного подъема для оптимизации направленности ФАР содержит новую процедуру возвращения в допустимую область посредством масштабирования вектора решения.
  \item Предложенный гибридный алгоритм дифференциальной эволюции отличается от известных ранее наличием процедуры адаптации штрафа, в которой учитывается возврат в допустимую область посредством масштабирования решения, что приводит к сокращению погрешности получаемых решений.
  \item Ранее при решении задач оптимизации направленности ФАР, как правило, не использовалась инвариантность основных свойств решений относительно равного сдвига фаз во всех излучателях. Однако, как показано в настоящей работе, учет такой инвариантности позволяет снизить размерность задачи и в результате сократить среднее время счета некоторых решателей.
  \item Впервые для задачи оптимизации направленности ФАР  показано наличие кластеров из локальных оптимумов с одинаковым значением целевой функции, и не эквивалентных относительно равного сдвига фаз во всех излучателях.
  \item Впервые обоснована целесообразность учета взаимного влияния излучателей при оптимизации направленности ФАР КВ диапазона.
\end{enumerate}

%Впервые показана нецелесообразность использования широкополосных вертикальных излучателей в составе ФАР коротковолнового диапазона, что следует из проведенных в работе расчетов по оптимизации ФАР с такими излучающими элементами.

{\influence}.
%ФАР КВ диапазона не так широко распространены ввиду невозможности нивелировать взаимное влияние излучателей, что существенно осложняет оптимизацию их направленности.
Разработанные алгоритмы оптимизации возбуждения ФАР могут  применяться в системах связи коротковолнового диапазона для увеличения дальности, снижения энергозатрат или площади, занимаемой антеннами. Созданное программное обеспечение позволяет производить необходимые для этого расчеты.
Полученное обоснование необходимости учета взаимного влияния излучателей при оптимизации напраленности ФАР, а также результаты вычислительных экспериментов для различных вариантов ФАР могут быть полезны при проектировании новых антенных систем.
Практическая значимость результатов исследования при выполнении работ по антенной тематике 
подтверждена в АО «Омский научно-исследовательский институт приборостроения».

{\theorInfluence}.
Осуществленный в работе переход от задачи оптимизации направленности ФАР в комплексных числах к задаче математического программирования позволил переформулировать в терминах математического программирования известные физические свойства задачи, в частности, инвариантность относительно сдвига фаз и закон сохранения энергии. Благодаря такому переносу, представленные в работе примеры симметрий задачи были теоретически обобщены на более широкий класс задач квадратичного программирования с использованием теории групп Ли в работах других авторов. Предложенная процедура возврата в допустимую область с помощью масштабирования вектора решения, а также построенная верхняя оценка эвклидовой нормы допустимых решений могут быть использованы при разработке новых методов математической оптимизации для задач, аналогичных рассматриваемой. Результаты диссертации используются в учебном процессе в ФГАОУ ВО «Омский государственный университет
им. Ф.М. Достоевского».


{\methods}
\ifnumequal{\value{myspec}}{1}{
В данной работе мы рассматриваем подход к решению задачи максимизации направленности излучения ФАР в заданном направлении при ограничениях, накладываемых на мощность, подаваемую на каждый из излучателей. Такая задача может быть решена только численными методами\mycite{yurkov:farkv}{Юрков~А.С.,~2014}. Для использования градиентного метода задача сводится к задаче безусловной оптимизации методом штрафных функций.
%Выбор градиентного алгоритма связан с тем, что отыскание даже локального оптимума в задаче невыпуклого квадратичного программирования может представлять собой NP-трудную задачу, и одним из методов, уместных в таких случаях, является градиентный алгоритм\autocite{murty:np}. Согласно\autocite{nesterov:nonconvex}, использование метода сопряженных градиентов для решения данной задачи не будет приводить к существенным улучшениям по сравнению с простым градиентным подъемом. Данное утверждение нашло согласие с результатами предварительных вычислительных экспериментов, проведенных нами для некоторых из рассматриваемых задач.

Вообще говоря, при использовании метода градиентного подъема не гарантируется получение глобального оптимума. Приблизиться к глобальному
оптимуму позволяет многократный запуск алгоритма из случайным образом сгенерированных точек. Кроме того, многократный запуск позволяет
оценить количество локальных оптимумов, что является некоторым критерием сложности индивидуальной задачи\mycite{eremeev:confidence}{Еремеев~А.В.,~2017}. Анализ структуры локальных оптимумов позволяет также выявить наличие нетривиальных симметрий.

Еще одним широко используемым подоходом к решению задач оптимизации ФАР являются эволюционные алгоритмы, и, в частности, генетические алгоритмы, роевые алгоритмы, алгоритмы дифференциальной эволюции\mycite{fanyaev2017, indenbometal:synthesis, rao:synthesis}{Indenbom~M.,~2017, Фаняев~И.А.,~2017, Rao~A.,~2017}.
%~\cite{storn:de,noguchi:de}.
Использование эволюционных методов требует больше времени, чем использование градиентного подъема, однако, в отличие от градиентных методов, не требует вычисления производных и не подвержен преждевременному завершению в точках стационарности.
%Таким образом, методы ДЭ также могут быть применены при исследовании структуры локальных оптимумов задачи невыпуклого квадратичного программирования.

Для оценки качества результатов градиентного алгоритма производится их сравнение с решениями, полученными с помощью решателя
BARON в пакете GAMS. BARON использует алгоритмы метода ветвей и границ, усиленные различными методами распространения ограничений и двойственности для уменьшения диапазонов переменных в ходе работы алгоритма\mycite{ryoo:nlp}{Ryoo~H.S.,~1995}. Его использование также представляет альтернативный подход к решению данной задачи, но, поскольку BARON является коммерческим решателем, произведение расчетов требует приобретения лицензии, что не всегда приемлемо.

}{
    Методологической базой являются работы по методам математической оптимизации. Использованы методы системного анализа, теории оптимизации, математического моделирования антенн и прикладной статистики.
}


{\defpositions}
\begin{enumerate}[beginpenalty=10000] % https://tex.stackexchange.com/a/476052/104425
%  \item В большинстве случаев, коммерческий решатель показывает лучшие результаты по сравнению с градиентными методами.
  \item Группа непрерывных симметрий рассматриваемой задачи одномерна и ее элементы соответствуют сдвигу фаз во всех излучателях на равную величину, что позволяет снизить размерность задачи на одну переменную и сократить время счета.
  \item Имеется интервал параметров кольцевых ФАР, в котором учет взаимного влияния излучателей ведет к существенному увеличению коэффициента усиления в заданном направлении.
  \item Для большинства рассмотренных конфигураций ФАР задача имеет несколько кластеров из локальных оптимумов с одинаковым значением целевой функции, не эквивалентных относительно равного сдвига фаз во всех излучателях.
  \item Использование градиентных методов в комбинации с методом ДЭ позволяет достичь конкурентоспособных решений по сравнению с алгоритмом ветвей и границ в задаче оптимизации фаз и амплитуд ФАР, особенно на задачах большой размерности.
\end{enumerate}
%В папке Documents можно ознакомиться с решением совета из Томского~ГУ
%(в~файле \verb+Def_positions.pdf+), где обоснованно даются рекомендации
%по~формулировкам защищаемых положений.

{\reliability} полученных результатов подтверждается согласованностью  результатов, полученных автором с использованием предложенных алгоритмов, с результатами коммерческого решателя BARON. Кроме того, достоверность подтверждается проведенными исследованиями адекватности модели с точки зрения физических принципов.


{\probation}
Основные результаты работы докладывались~на:
\begin{enumerate}[beginpenalty=10000] % https://tex.stackexchange.com/a/476052/104425
  \item Международной конференции <<Теория математической оптимизации и исследование операций (МОТОР)>> - Петрозаводск, июль, 2022.
  \item Международной конференции <<Теория математической оптимизации и исследование операций (МОТОР)>> -  Иркутск, июль 2021.
  \item Семинаре <<Современные проблемы радиофизики и радиотехники>>, Омск, 2021.
  \item Международной конференции <<Теория математической оптимизации и исследование операций (МОТОР)>> - Екатеринбург, июль 2019.
  \item V Международной научно-технической конференции <<Радиотехника, электроника и связь>> - Омск, октябрь 2019.
  \item VII Международной конференции <<Проблемы оптимизации и их приложения>> - Омск, июль 2018.
  \item Семинаре <<Математическое моделирование и дискретная оптимизация>>, Омск, 2018 --- 2022.
  \item Семинаре <<Перспективы развития радиосвязи и приборостроения>>, Омск, 2018. 
\end{enumerate}

{\contribution} Автор адаптировал метод градиентного подъема и алгоритм дифференциальной эволюции к специфике рассматриваемой задачи, осуществил переход к задаче квадратичного программирования, исходя из постановки в комплексных числах, исследовал наличие непрерывных симметрий, проводил вычислительные эксперименты, исследовал устойчивость решений к возмущению исходных данных, формулировал выводы.
Основные результаты диссертационного исследования получены в ходе выполнения гранта РФФИ \No~19-37-90066 по конкурсу <<Аспиранты>> в 2019-2021 гг. в Омском филиале Института математики им. С.Л. Соболева СО РАН. 

\ifnumequal{\value{bibliosel}}{0}
{%%% Встроенная реализация с загрузкой файла через движок bibtex8. (При желании, внутри можно использовать обычные ссылки, наподобие `\cite{vakbib1,vakbib2}`).
    {\publications} Основные результаты по теме диссертации изложены
    в~9~печатных изданиях,
    3 из которых изданы в журналах, рекомендованных ВАК,
    6 "--- в тезисах докладов. Зарегистрирована 1 программа для ЭВМ.
}%
{%%% Реализация пакетом biblatex через движок biber
    \begin{refsection}[bl-author, bl-registered]
        % Это refsection=1.
        % Процитированные здесь работы:
        %  * подсчитываются, для автоматического составления фразы "Основные результаты ..."
        %  * попадают в авторскую библиографию, при usefootcite==0 и стиле `\insertbiblioauthor` или `\insertbiblioauthorgrouped`
        %  * нумеруются там в зависимости от порядка команд `\printbibliography` в этом разделе.
        %  * при использовании `\insertbiblioauthorgrouped`, порядок команд `\printbibliography` в нём должен быть тем же (см. biblio/biblatex.tex)
        %
        % Невидимый библиографический список для подсчёта количества публикаций:
        \printbibliography[heading=nobibheading, section=1, env=countauthorvak,          keyword=biblioauthorvak]%
        \printbibliography[heading=nobibheading, section=1, env=countauthorwos,          keyword=biblioauthorwos]%
        \printbibliography[heading=nobibheading, section=1, env=countauthorscopus,       keyword=biblioauthorscopus]%
        \printbibliography[heading=nobibheading, section=1, env=countauthorconf,         keyword=biblioauthorconf]%
        \printbibliography[heading=nobibheading, section=1, env=countauthorother,        keyword=biblioauthorother]%
        \printbibliography[heading=nobibheading, section=1, env=countregistered,         keyword=biblioregistered]%
        \printbibliography[heading=nobibheading, section=1, env=countauthorpatent,       keyword=biblioauthorpatent]%
        \printbibliography[heading=nobibheading, section=1, env=countauthorprogram,      keyword=biblioauthorprogram]%
        \printbibliography[heading=nobibheading, section=1, env=countauthor,             keyword=biblioauthor]%
        \printbibliography[heading=nobibheading, section=1, env=countauthorvakscopuswos, filter=vakscopuswos]%
        \printbibliography[heading=nobibheading, section=1, env=countauthorscopuswos,    filter=scopuswos]%
        %
        \nocite{*}%
        %
        {\publications} Основные результаты по теме диссертации изложены в \arabic{citeauthor} печатных изданиях,
        \arabic{citeauthorvak} из которых изданы в журналах, рекомендованных ВАК или прираненных к ним\sloppy%
        \ifnum \value{citeauthorscopuswos}>0%
            , \arabic{citeauthorscopuswos} "--- в~периодических научных журналах, индексируемых Web of~Science и Scopus\sloppy%
        \fi%
        \ifnum \value{citeauthorconf}>0%
            , \arabic{citeauthorconf} "--- в~тезисах докладов.
        \else%
            .
        \fi%
        \ifnum \value{citeregistered}=1%
            \ifnum \value{citeauthorpatent}=1%
                Зарегистрирован \arabic{citeauthorpatent} патент.
            \fi%
            \ifnum \value{citeauthorprogram}=1%
                Зарегистрирована \arabic{citeauthorprogram} программа для ЭВМ.
            \fi%
        \fi%
        \ifnum \value{citeregistered}>1%
            Зарегистрированы\ %
            \ifnum \value{citeauthorpatent}>0%
            \formbytotal{citeauthorpatent}{патент}{}{а}{}\sloppy%
            \ifnum \value{citeauthorprogram}=0 . \else \ и~\fi%
            \fi%
            \ifnum \value{citeauthorprogram}>0%
            \formbytotal{citeauthorprogram}{программ}{а}{ы}{} для ЭВМ.
            \fi%
        \fi%
        % К публикациям, в которых излагаются основные научные результаты диссертации на соискание учёной
        % степени, в рецензируемых изданиях приравниваются патенты на изобретения, патенты (свидетельства) на
        % полезную модель, патенты на промышленный образец, патенты на селекционные достижения, свидетельства
        % на программу для электронных вычислительных машин, базу данных, топологию интегральных микросхем,
        % зарегистрированные в установленном порядке.(в ред. Постановления Правительства РФ от 21.04.2016 N 335)
    \end{refsection}%
    \begin{refsection}[bl-author, bl-registered]
        % Это refsection=2.
        % Процитированные здесь работы:
        %  * попадают в авторскую библиографию, при usefootcite==0 и стиле `\insertbiblioauthorimportant`.
        %  * ни на что не влияют в противном случае
        \nocite{tyu:daor}%
        \nocite{tyu:jphys}%
        \nocite{tyu:motor}%
        \nocite{tyu:msim22}%
        \nocite{tyu:opta}%
        \nocite{tyu:reis}%
        \nocite{tyu22:ring}%
        \nocite{tyu:fmh}%
    \end{refsection}%
        %
        % Всё, что вне этих двух refsection, это refsection=0,
        %  * для диссертации - это нормальные ссылки, попадающие в обычную библиографию
        %  * для автореферата:
        %     * при usefootcite==0, ссылка корректно сработает только для источника из `external.bib`. Для своих работ --- напечатает "[0]" (и даже Warning не вылезет).
        %     * при usefootcite==1, ссылка сработает нормально. В авторской библиографии будут только процитированные в refsection=0 работы.
}
