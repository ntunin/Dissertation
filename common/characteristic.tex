
{\actuality}
\ifnumequal{\value{myspec}}{0}{
    Актуальность для 05.13.18
}{
В настоящее время разработка и анализ эффективных систем радиосвязи имеет большое значение для народного хозяйства. Одной из актуальных задач в этой области является задача оптимизации направленности фазированных антенных решеток (ФАР), представляющих собой антенные системы, распределение фаз и амплитуд на излучателях которых позволяет получать направленное излучение. 
%Интенсивность излучения каждого отдельно взятого излучателя такой системы зависит от геометрии самого излучателя и радиочастоты сигнала, поданного на его вход. Характеристика интенсивности излучения антенны в пространстве называется диаграммой направленности. 
Будучи собранными в антенную систему и разведенными в пространстве, излучатели формируют диаграмму направленности, которая зависит от расположения и конструкции излучателей, а также выбора фаз и амплитуд сигналов, подаваемых на вход излучателей. 
%Это связано с тем, что часть излучения итерферирует с собственным излучением других излучателей, а часть отражается от них и также вносит вклад в результирующее излучение. Таким образом, фазируя определенным образом сигналы, подаваемые на вход каждого излучателя, можно получить максимум интенсивности излучения в заданном направлении. 
В диапазоне сверх высоких частот (СВЧ) задачи оптимизации направленности ФАР успешно решаются благодаря допустимости упрощающего предположения о
малом взаимном влиянии излучателей. Однако, в диапазоне высоких частот (ВЧ), также называемом коротковолновым, требуется использовать методы электродинамического моделирования для оценки наведенных токов для каждой пары излучателей, в результате задача оптимизации направленности ФАР оказывается более сложной, и потому мало изучена.
    Диссертационная работа посвящена исследованию и разработке методов оптимизации направленного излучения ФАР ВЧ диапазона с целью увеличения
дальности передачи радиосигнала в коротковолновом диапазоне и сокращения расходов энергии и площади, занимаемой ФАР.

}


% {\progress}
% Этот раздел должен быть отдельным структурным элементом по
% ГОСТ, но он, как правило, включается в описание актуальности
% темы. Нужен он отдельным структурынм элемементом или нет ---
% смотрите другие диссертации вашего совета, скорее всего не нужен.

{\aim} данной работы является исследование методов оптимизации для обеспечения направленного излучения фазированных антенных решеток коротковолнового диапазона.

Для~достижения поставленной цели необходимо было решить следующие {\tasks}:
\begin{enumerate}[beginpenalty=10000] % https://tex.stackexchange.com/a/476052/104425
  \item Разработать реализации алгоритмов для решения задачи оптимизации направленности ФАР КВ диапазона и сравнить их результаты с результатами коммерческих решателей.
  \item Исследовать целесообразность учета взаимного влияния излучателей при оптимизации направленности ФАР КВ диапазона.
  \item Разработать программный комплекс для построения и решения задач оптимизации направленного излучения ФАР.
\end{enumerate}


{\novelty}
Использование ФАР широко распространено в диапазоне сверхвысоких частот (СВЧ), поскольку к
достоинствам данного диапазона относятся малые размеры антенн и более широкая абсолютная полоса частот. Однако, радиосистемы СВЧ диапазона характеризуются малой дальностью действия. ФАР высокочастотного диапазона (ВЧ), также называемого коротковолновым (КВ), менее распространены ввиду больших габаритов, однако, возможность увеличения дальности действия привлекает к ним особый интерес~\cite{kudzinetal:compact,wilensky:highpower,yalanetal:design}.

Рассматриваемая в статье задача состоит в максимизации направленности излучения ФАР за счет выбора фаз и амплитуд в каждом излучателе. Задача осложняется за счет взаимного влияния излучателей друг на друга. В СВЧ диапазоне такое влияние можно нивелировать с помощью дополнительных устройств~(см.~\cite{sazonov:PAA}, \S~6.7), после чего задача существенно упрощается~\cite{indenbometal:synthesis}. В случае же с ВЧ диапазоном, в котором используются иначе спроектированные излучатели, такое допущение невозможно.

Известны следующие постановки задачи оптимизации направленности ФАР КВ диапазона с учетом взаимного влияния излучателей~\cite{yurkov:farkv}: при ограничении суммарной мощности, подаваемой на антенную систему, и при ограничении на мощность по каждому входу антенной системы. В случае с ограничением на суммарную мощность задача может быть решена аналитически.
В данной работе разработаны методы решения задачи оптимизации направленности ФАР КВ диапазона с учетом взаимного влияния излучателей при ограничениях на мощность по каждому входу антенной системы.

{\influence}. ФАР КВ диапазона не так широко распространены ввиду невозможности нивелировать взаимное влияние излучателей, что существенно осложняет оптимизацию их направленности. В рамках данной работы были получены важные практические результаты, обосновывающие необходимость учета взаимного влияния излучателей при оптимизации напраленности ФАР. Кроме того, в работе демонстрируются подходы и техники, позволяющие оптимизировать направленность антенной системы с учетом взаимного влияния излучателей, а также разработан программный комплекс, позволяющий производить все необходимые расчеты.

{\methods}

\ifnumequal{\value{myspec}}{0}{
В данной работе мы рассматриваем подход к решению задачи максимизации направленности излучения ФАР в заданном направлении при ограничениях, накладываемых на мощность, подаваемую на каждый из излучателей. Такая задача может быть решена только численными методами~\cite{yurkov:farkv}. Для использования градиентного метода задача сводится к задаче безусловной оптимизации методом штрафных функций. Выбор градиентного алгоритма связан с тем, что отыскание даже локального оптимума в задаче невыпуклого квадратичного программирования может представлять собой NP-трудную задачу, и одним из методов, уместных в таких случаях, является градиентный алгоритм~\cite{murty:np}. Согласно~\cite{nesterov:nonconvex}, использование метода сопряженных градиентов для решения данной задачи не будет приводить к существенным улучшениям по сравнению с простым градиентным подъемом. Данное утверждение нашло согласие с результатами предварительных вычислительных экспериментов, проведенных нами для некоторых из рассматриваемых задач.

Для оценки качества результатов градиентного алгоритма производится их сравнение с решениями, полученными с помощью решателя
BARON в пакете GAMS. BARON использует алгоритмы метода ветвей и границ, усиленные различными методами распространения ограничений и двойственности для уменьшения диапазонов переменных в ходе работы алгоритма~\cite{ryoo:nlp}. Его использование также представляет альтернативный подход к решению данной задачи, но, поскольку BARON является коммерческим решателем, произведение расчетов требует приобретения лицензии, что не всегда приемлемо.

Еще одним эффективным подоходом к решению невыпуклых задач квадратичного программирования являются генетические алгоритмы, и, в частности, методы дифференциальной эволюции (ДЭ)~\cite{storn:de,noguchi:de}. Использование методов ДЭ требует больше времени, чем использование градиентного подъема, однако, в отличие от градиентных методов, не требует вычисления производных и не подвержен преждевременному завершению в точках стационарности. Таким образом, методы ДЭ также могут быть применены при исследовании структуры локальных оптимумов задачи невыпуклого квадратичного программирования.

Вообще говоря, при использовании метода градиентного подъема не гарантируется получение глобального оптимума. Приблизиться к глобальному
оптимуму позволяет многократный запуск алгоритма из случайным образом сгенерированных точек. Кроме того, многократный запуск позволяет
оценить количество локальных оптимумов, что является некоторым критерием сложности индивидуальной задачи~\cite{eremeev:confidence}. Анализ структуры локальных оптимумов позволяет также выявить наличие нетривиальных симметрий.
}{
    Методологической базой являются работы по методам математической оптимизации. Использованы методы системного анализа, исследования операций, теории оптимизации и прикладной статистики.
}


{\defpositions}
\begin{enumerate}[beginpenalty=10000] % https://tex.stackexchange.com/a/476052/104425
  \item В большинстве случаев, коммерческий решатель показывает лучшие результаты по сравнению с градиентными методами.
  \item Использование алгоритмов дифференциальной эволюции позволяет достичь более качественных решений по сравнению с градиентными методами.
  \item Проведены вычислительные эксперименты, выявляющие наличие непрерывных групп симметрий допустимых решений. Для всех рассмотренных задач выявлено наличие только фазовой симметрии.
  \item Учет взаимного влияния излучателей друг на друга может оказать существенное влияние на результирующее излучение.
  \item Выявлены конфигурации решеток, для которых учет взаимного влияния ведет к существенному увеличению коэффициента усиления.
\end{enumerate}
%В папке Documents можно ознакомиться с решением совета из Томского~ГУ
%(в~файле \verb+Def_positions.pdf+), где обоснованно даются рекомендации
%по~формулировкам защищаемых положений.

{\reliability} полученных результатов обеспечивается их соответствием с результатами коммерческих решателей, а также проведенными исследованиями адекватности модели в сравнении с общими физическими принципами.


{\probation}
Основные результаты работы докладывались~на:
\begin{enumerate}[beginpenalty=10000] % https://tex.stackexchange.com/a/476052/104425
  \item VII Международной конференции «Проблемы оптимизации и их приложения» - Омск, июль 2018.
  \item Международной конференции «Теория математической оптимизации и исследование операций» - Екатеринбург, июль 2019.
  \item V Международной научно-технической конференции «Радиотехника, электроника и связь» - Омск, октябрь 2019.
  \item Международная конференция «Теория математической оптимизации и исследование операций» - Иркутск, июль 2022.
\end{enumerate}

{\contribution} Автор принимал активное участие в разработке программного комплекса, разработке эволюционных алгоритмов, адаптации метода градиентного подъема к специфике рассматриваемой задачи, проведении вычислительных экспериментов, анализе и интерпретации их результатов.

\ifnumequal{\value{bibliosel}}{0}
{%%% Встроенная реализация с загрузкой файла через движок bibtex8. (При желании, внутри можно использовать обычные ссылки, наподобие `\cite{vakbib1,vakbib2}`).
    {\publications} Основные результаты по теме диссертации изложены
    в~7~печатных изданиях,
    3 из которых изданы в журналах, рекомендованных ВАК,
    3 "--- в тезисах докладов.
}%
{%%% Реализация пакетом biblatex через движок biber
    \begin{refsection}[bl-author, bl-registered]
        % Это refsection=1.
        % Процитированные здесь работы:
        %  * подсчитываются, для автоматического составления фразы "Основные результаты ..."
        %  * попадают в авторскую библиографию, при usefootcite==0 и стиле `\insertbiblioauthor` или `\insertbiblioauthorgrouped`
        %  * нумеруются там в зависимости от порядка команд `\printbibliography` в этом разделе.
        %  * при использовании `\insertbiblioauthorgrouped`, порядок команд `\printbibliography` в нём должен быть тем же (см. biblio/biblatex.tex)
        %
        % Невидимый библиографический список для подсчёта количества публикаций:
        \printbibliography[heading=nobibheading, section=1, env=countauthorvak,          keyword=biblioauthorvak]%
        \printbibliography[heading=nobibheading, section=1, env=countauthorwos,          keyword=biblioauthorwos]%
        \printbibliography[heading=nobibheading, section=1, env=countauthorscopus,       keyword=biblioauthorscopus]%
        \printbibliography[heading=nobibheading, section=1, env=countauthorconf,         keyword=biblioauthorconf]%
        \printbibliography[heading=nobibheading, section=1, env=countauthorother,        keyword=biblioauthorother]%
        \printbibliography[heading=nobibheading, section=1, env=countregistered,         keyword=biblioregistered]%
        \printbibliography[heading=nobibheading, section=1, env=countauthorpatent,       keyword=biblioauthorpatent]%
        \printbibliography[heading=nobibheading, section=1, env=countauthorprogram,      keyword=biblioauthorprogram]%
        \printbibliography[heading=nobibheading, section=1, env=countauthor,             keyword=biblioauthor]%
        \printbibliography[heading=nobibheading, section=1, env=countauthorvakscopuswos, filter=vakscopuswos]%
        \printbibliography[heading=nobibheading, section=1, env=countauthorscopuswos,    filter=scopuswos]%
        %
        \nocite{*}%
        %
        {\publications} Основные результаты по теме диссертации изложены в~\arabic{citeauthor}~печатных изданиях,
        \arabic{citeauthorvak} из которых изданы в журналах, рекомендованных ВАК\sloppy%
        \ifnum \value{citeauthorscopuswos}>0%
            , \arabic{citeauthorscopuswos} "--- в~периодических научных журналах, индексируемых Web of~Science и Scopus\sloppy%
        \fi%
        \ifnum \value{citeauthorconf}>0%
            , \arabic{citeauthorconf} "--- в~тезисах докладов.
        \else%
            .
        \fi%
        \ifnum \value{citeregistered}=1%
            \ifnum \value{citeauthorpatent}=1%
                Зарегистрирован \arabic{citeauthorpatent} патент.
            \fi%
            \ifnum \value{citeauthorprogram}=1%
                Зарегистрирована \arabic{citeauthorprogram} программа для ЭВМ.
            \fi%
        \fi%
        \ifnum \value{citeregistered}>1%
            Зарегистрированы\ %
            \ifnum \value{citeauthorpatent}>0%
            \formbytotal{citeauthorpatent}{патент}{}{а}{}\sloppy%
            \ifnum \value{citeauthorprogram}=0 . \else \ и~\fi%
            \fi%
            \ifnum \value{citeauthorprogram}>0%
            \formbytotal{citeauthorprogram}{программ}{а}{ы}{} для ЭВМ.
            \fi%
        \fi%
        % К публикациям, в которых излагаются основные научные результаты диссертации на соискание учёной
        % степени, в рецензируемых изданиях приравниваются патенты на изобретения, патенты (свидетельства) на
        % полезную модель, патенты на промышленный образец, патенты на селекционные достижения, свидетельства
        % на программу для электронных вычислительных машин, базу данных, топологию интегральных микросхем,
        % зарегистрированные в установленном порядке.(в ред. Постановления Правительства РФ от 21.04.2016 N 335)
    \end{refsection}%
    \begin{refsection}[bl-author, bl-registered]
        % Это refsection=2.
        % Процитированные здесь работы:
        %  * попадают в авторскую библиографию, при usefootcite==0 и стиле `\insertbiblioauthorimportant`.
        %  * ни на что не влияют в противном случае
        \nocite{tyu:daor}%
        \nocite{tyu:jphys}%
        \nocite{tyu:motor}%
        \nocite{tyu:opta}%
        \nocite{tyu:reis}%
        \nocite{tyu:fmh}%
    \end{refsection}%
        %
        % Всё, что вне этих двух refsection, это refsection=0,
        %  * для диссертации - это нормальные ссылки, попадающие в обычную библиографию
        %  * для автореферата:
        %     * при usefootcite==0, ссылка корректно сработает только для источника из `external.bib`. Для своих работ --- напечатает "[0]" (и даже Warning не вылезет).
        %     * при usefootcite==1, ссылка сработает нормально. В авторской библиографии будут только процитированные в refsection=0 работы.
}
