%% Согласно ГОСТ Р 7.0.11-2011:
%% 5.3.3 В заключении диссертации излагают итоги выполненного исследования, рекомендации, перспективы дальнейшей разработки темы.
%% 9.2.3 В заключении автореферата диссертации излагают итоги данного исследования, рекомендации и перспективы дальнейшей разработки темы.
\begin{enumerate}
  \item В ходе вычислительного эксперимента показано, что задача оптимизации направленности фазированной антенной решетки имеет многочисленные локальные оптимумы, большое число из которых совпадают по целевой функции, однако не эквивалентны между собой относительно равного сдвига фаз во всех излучателях.
  \item Установлено, что непрерывная подгруппа линейных симметрий для рассматриваемых задач оптимизации направленности фазированной антенной решетки одномерна и ее элементы соответствуют сдвигу фаз во всех излучателях на равную величину, что позволяет снизить размерность задачи на одну переменную и сократить время счета коммерческого решателя BARON.
 \item Предложена модификация алгоритма дифференциальной эволюции в комбинации с градиентным алгоритмом, учитывающая специфику задачи оптимизации направленности фазированной антенной решетки, и показавшая свою конкурентоспособность по сравнению с коммерческим решателем BARON с преимуществом на задачах большей размерности.
  \item В результате вычислительного эксперимента по исследованию влияния расположения излучателей и используемой радиочастоты на эффективность работы алгоритмов оптимизации было обнаружено, что имеются конфигурации, при которых усиление ФАР существенно превосходит усиление одиночного излучателя.
%В текущей работе была рассмотрена постановка задачи оптимизации направленности излучения антенной системы, представленной в виде регулярной решетки излучателей. Для данной задачи была разработана модель квадратичного программирования в вещественных числах. Произведено сравнение результатов разработанных алгоритмов в вычислительном эксперименте.
  \item Выявлены ситуации, в которых коэффициент усиления, соответствующий решению задачи квадратичной оптимизации, имеет существенное преимущество (до 5 дб) перед коэффициентом усиления, получаемым стандартным методом фазирования без учета взаимного влияния.
\end{enumerate}
