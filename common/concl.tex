%% Согласно ГОСТ Р 7.0.11-2011:
%% 5.3.3 В заключении диссертации излагают итоги выполненного исследования, рекомендации, перспективы дальнейшей разработки темы.
%% 9.2.3 В заключении автореферата диссертации излагают итоги данного исследования, рекомендации и перспективы дальнейшей разработки темы.
\begin{enumerate}
%\item Предложена модификация градиентного метода, учитывающая специфику задачи оптимизации фаз и амплитуд ФАР. Решения, полученные этим алгоритмом на тестовых задачах, уступали результатов метода ветвей и границ с локальным поиском из коммерческого решателя не более, чем на 1\% по целевой функции.
\item Предложена многопроцессорная модификация алгоритма дифференциальной эволюции в комбинации с градиентным алгоритмом. Решения, полученные этим алгоритмом, на тестовых задачах большой размерности оказались точнее, чем результаты коммерческого решателя BARON.   
  \item Методами линейной алгебры выявлена одномерная непрерывная группа симметрий в задаче оптимизации фаз и амплитуд ФАР. Все тестовые задачи оказались симметричны только относительно равного сдвига фаз во всех излучателях.
  \item В ходе вычислительного эксперимента показано, что задача оптимизации фаз и амплитуд фазированной антенной решетки имеет многочисленные локальные оптимумы, многие из которых совпадают по целевой функции, однако не эквивалентны между собой относительно равного сдвига фаз во всех излучателях.
%В текущей работе была рассмотрена постановка задачи оптимизации направленности излучения антенной системы, представленной в виде регулярной решетки излучателей. Для данной задачи была разработана модель квадратичного программирования в вещественных числах. Произведено сравнение результатов разработанных алгоритмов в вычислительном эксперименте.
  \item Выявлены ситуации, в которых коэффициент усиления, соответствующий решению задачи квадратичной оптимизации, имеет существенное преимущество (до 5 дб) перед коэффициентом усиления, получаемым стандартным методом простого фазирования. 
\end{enumerate}
