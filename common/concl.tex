%% Согласно ГОСТ Р 7.0.11-2011:
%% 5.3.3 В заключении диссертации излагают итоги выполненного исследования, рекомендации, перспективы дальнейшей разработки темы.
%% 9.2.3 В заключении автореферата диссертации излагают итоги данного исследования, рекомендации и перспективы дальнейшей разработки темы.
\begin{enumerate}
  \item В текущей работе была рассмотрена постановка задачи оптимизации направленности излучения антенной системы, представленной в виде регулярной решетки излучателей. Для данной задачи была разработана модель квадратичного программирования в вещественных числах. Произведено сравнение результатов разработанных алгоритмов в вычислительном эксперименте.
  \item Проведены вычислительные эксперименты, выявляющие наличие непрерывных групп симметрий допустимых решений. Для всех рассмотренных задач выявлено наличие только фазовой симметрии.
  \item В рамках данного исследования было выявлено наличие ситуаций, в которых коэффициент усиления, соответствующий решению задачи математического программирования, имеет существенное преимущество перед коэффициентом усиления простого фазирования. Выявлено, что, при оптимизации направленности ФАР КВ диапазона целесообразны расчеты с учетом взаимного влияния излучателей.
  \item Для генерации тестовых примеров, автоматизации вычислительных экспериментов и визуализации их результатов были разработаны интерпретатор <<Expi>> и его графическая оболочка <<ExpiIDE>>.
\end{enumerate}
